\documentclass{article}
\usepackage[utf8]{inputenc} % Wsparcie dla UTF-8
\usepackage[T1]{fontenc}    % Kodowanie T1 dla polskich znaków
\usepackage{polski}         % Pakiet do języka polskiego
\usepackage{lmodern}        % Lepsza czcionka
\usepackage{amsmath}
\usepackage{amssymb}
\usepackage{listings}
\usepackage{xcolor}
\usepackage{geometry}
\geometry{a4paper, margin=1in}

\definecolor{codegreen}{rgb}{0,0.6,0}
\definecolor{codegray}{rgb}{0.5,0.5,0.5}
\definecolor{codepurple}{rgb}{0.58,0,0.82}
\lstset{
    basicstyle=\ttfamily\small,
    commentstyle=\color{codegreen},
    keywordstyle=\color{magenta},
    numberstyle=\tiny\color{codegray},
    stringstyle=\color{codepurple},
    breakatwhitespace=false,
    breaklines=true,
    captionpos=b,
    keepspaces=true,
    numbers=left,
    numbersep=5pt,
    showspaces=false,
    showstringspaces=false,
    showtabs=false,
    tabsize=2
}

\title{Interpolacja Newtona}
\author{Student}
\date{}

\begin{document}

\maketitle

\section{Podstawy teoretyczne}
Dla $n+1$ punktów $(x_0, y_0), (x_1, y_1), \dots, (x_n, y_n)$, gdzie $x_i$ są różne, wielomian interpolacyjny w \textbf{postaci Newtona} ma postać:
\[
P_n(x) = a_0 + a_1(x - x_0) + a_2(x - x_0)(x - x_1) + \dots + a_n(x - x_0)\cdots(x - x_{n-1}),
\]
gdzie:
- $a_k = f[x_0, x_1, \dots, x_k]$ to \textbf{ilorazy różnicowe} rzędu $k$,
- Ilorazy obliczane są rekurencyjnie:
  \[
  f[x_i, \dots, x_j] = \frac{f[x_{i+1}, \dots, x_j] - f[x_i, \dots, x_{j-1}]}{x_j - x_i}.
  \]

\section{Kod w Pythonie}
\subsection{Pełny kod}
\begin{lstlisting}[language=Python, caption=Interpolacja Newtona]
import sys

def main():
    data = sys.stdin.read().splitlines()
    if not data:
        return
    
    n = int(data[0].strip())
    x = list(map(float, data[1].split()))
    y = list(map(float, data[2].split()))
    t_values = []
    
    for line in data[3:]:
        t_values.extend(map(float, line.split()))
    
    m = n + 1
    coeffs = y.copy()
    
    for i in range(1, m):
        for j in range(m - 1, i - 1, -1):
            coeffs[j] = (coeffs[j] - coeffs[j - 1]) / (x[j] - x[j - i])
    
    results = []
    for t in t_values:
        result = coeffs[-1]
        for i in range(m - 2, -1, -1):
            result = coeffs[i] + (t - x[i]) * result
        results.append(result)
    
    for res in results:
        print(res)

if __name__ == "__main__":
    main()
\end{lstlisting}

\subsection{Kluczowe fragmenty}
\subsubsection*{Obliczanie ilorazów różnicowych}
\begin{lstlisting}[language=Python]
for i in range(1, m):
    for j in range(m - 1, i - 1, -1):
        coeffs[j] = (coeffs[j] - coeffs[j - 1]) / (x[j] - x[j - i])
\end{lstlisting}
\textbf{Dlaczego od tyłu?} Aby nie nadpisać danych używanych w kolejnych obliczeniach (np. $f[x_0, x_1]$ musi zostać zachowane do obliczenia $f[x_0, x_1, x_2]$).

\subsubsection*{Schemat Hornera dla postaci Newtona}
\begin{lstlisting}[language=Python]
result = coeffs[-1]
for i in range(m - 2, -1, -1):
    result = coeffs[i] + (t - x[i]) * result
\end{lstlisting}
\textbf{Działanie:} 
\[
P(t) = a_0 + (t - x_0)\big(a_1 + (t - x_1)\big(\dots\big)\big)
\]

\section{Przykład działania}
\subsection{Dane wejściowe}
\begin{lstlisting}
2
0.0 1.0 2.0
1.0 2.0 5.0
0.5 1.5
\end{lstlisting}

\subsection{Wynik}
\begin{lstlisting}
1.25
3.25
\end{lstlisting}

\textbf{Weryfikacja:}
- Dla $t = 0.5$: $P(0.5) = 1.0 + 0.5 \cdot (1.0 - 0.5 \cdot 1.0) = 1.25$,
- Dla $t = 1.5$: $P(1.5) = 1.0 + 1.5 \cdot (1.0 + 0.5 \cdot 1.0) = 3.25$.

\end{document}
