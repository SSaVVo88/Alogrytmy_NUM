\documentclass{article}
\usepackage{amsmath}
\usepackage{amssymb}
\usepackage{listings}
\usepackage{xcolor}
\usepackage{geometry}
\geometry{a4paper, margin=1in}

\definecolor{codegreen}{rgb}{0,0.6,0}
\definecolor{codegray}{rgb}{0.5,0.5,0.5}
\definecolor{codepurple}{rgb}{0.58,0,0.82}
\lstset{
    basicstyle=\ttfamily\small,
    commentstyle=\color{codegreen},
    keywordstyle=\color{magenta},
    numberstyle=\tiny\color{codegray},
    stringstyle=\color{codepurple},
    breakatwhitespace=false,
    breaklines=true,
    captionpos=b,
    keepspaces=true,
    numbers=left,
    numbersep=5pt,
    showspaces=false,
    showstringspaces=false,
    showtabs=false,
    tabsize=2
}

\title{Algorytmy Numeryczne 1.1 - Interpolacja Newtona}
\author{Przemyslaw Sawoniuk, Jeremiasz Olech}
\date{26.11.2025}

\begin{document}

\maketitle

\section{Podstawy teoretyczne}
Dla $n+1$ punktow $(x_0, y_0), (x_1, y_1), \dots, (x_n, y_n)$, gdzie $x_i$ sa rozne, wielomian interpolacyjny w \textbf{postaci Newtona} ma postac:
\[
P_n(x) = a_0 + a_1(x - x_0) + a_2(x - x_0)(x - x_1) + \dots + a_n(x - x_0)\cdots(x - x_{n-1}),
\]
gdzie:
\begin{itemize}
    \item $a_k = f[x_0, x_1, \dots, x_k]$ to \textbf{ilorazy roznicowe} rzedu $k$,
    \item Ilorazy obliczane sa rekurencyjnie:
    \[
    f[x_i, \dots, x_j] = \frac{f[x_{i+1}, \dots, x_j] - f[x_i, \dots, x_{j-1}]}{x_j - x_i}.
    \]
\end{itemize}

\section{Kod w Pythonie}
\subsection{Pelny kod}
\begin{lstlisting}[language=Python, caption=Interpolacja Newtona z obsluga bledow]
import sys
from validation import validate_and_parse

def main():
    print(
        "Podaj dane w nastepujacym formacie:\n"
        "n\n"
        "x0 x1 ... xn\n"
        "y0 y1 ... yn\n"
        "t1 t2 ... (opcjonalnie kolejne linie z t)\n\n"
        "Po zakonczeniu wpisywania danych nacisnij:\n"
        " - Ctrl+D  (Linux/Mac)\n"
        " - Ctrl+Z+Enter  (Windows)\n"
    )
    data = sys.stdin.read().splitlines()
    try:
        n, x, y, t_values = validate_and_parse(data)
    except ValueError as e:
        print("Blad:", e)
        return
    
    m = n + 1
    coeffs = y.copy()
    
    for i in range(1, m):
        for j in range(m - 1, i - 1, -1):
            coeffs[j] = (coeffs[j] - coeffs[j - 1]) / (x[j] - x[j - i])
    
    results = []
    for t in t_values:
        result = coeffs[-1]
        for i in range(m - 2, -1, -1):
            result = coeffs[i] + (t - x[i]) * result
        results.append(result)
    
    for res in results:
        print(res)

if __name__ == "__main__":
    main()
\end{lstlisting}

\subsection{Modul walidacji}
\begin{lstlisting}[language=Python, caption=validation.py]
def validate_and_parse(data):
    if not data:
        raise ValueError("Brak danych wejsciowych.")

    # n
    try:
        n = int(data[0].strip())
    except ValueError:
        raise ValueError("Pierwsza linia musi zawierac liczbe calkowita n.")

    if n < 1:
        raise ValueError("n musi byc >= 1.")

    if len(data) < 3:
        raise ValueError("Zbyt malo linii danych - oczekiwane: n, x, y.")

    # x
    try:
        x = list(map(float, data[1].split()))
    except ValueError:
        raise ValueError("Linia 2 musi zawierac wartosci liczbowe x.")

    if len(x) != n + 1:
        raise ValueError(f"Dla n={n} nalezy podac dokladnie {n+1} wartosci x (podano {len(x)}).")

    if len(set(x)) != len(x):
        raise ValueError("Wartosci x musza byc unikalne.")

    # y
    try:
        y = list(map(float, data[2].split()))
    except ValueError:
        raise ValueError("Linia 3 musi zawierac wartosci liczbowe y.")

    if len(y) != n + 1:
        raise ValueError(f"Dla n={n} nalezy podac dokladnie {n+1} wartosci y (podano {len(y)}).")

    # t
    t_values = []
    for line in data[3:]:
        try:
            t_values.extend(map(float, line.split()))
        except ValueError:
            raise ValueError("W liniach z t musza znajdowac sie liczby.")

    if not t_values:
        raise ValueError("Brak punktow t do obliczenia interpolacji.")

    return n, x, y, t_values
\end{lstlisting}

\section{Obsluga bledow}
\subsection{Kluczowe mechanizmy walidacji}
\begin{enumerate}
    \item \textbf{Sprawdzenie podstawowych warunkow:}
    \begin{itemize}
        \item Dane wejsciowe nie moga byc puste
        \item Liczba $n$ musi byc calkowita i $\geq 1$
        \item Musza byc podane wszystkie linie: $n$, $x$, $y$
    \end{itemize}
    
    \item \textbf{Walidacja wezlow $x$:}
    \begin{itemize}
        \item Wartosci musza byc liczbowe
        \item Liczba wezlow musi wynosic dokladnie $n+1$
        \item Wezly musza byc unikalne (brak powtorzen)
    \end{itemize}
    
    \item \textbf{Walidacja wartosci $y$:}
    \begin{itemize}
        \item Wartosci musza byc liczbowe
        \item Liczba wartosci musi wynosic dokladnie $n+1$
    \end{itemize}
    
    \item \textbf{Walidacja punktow $t$:}
    \begin{itemize}
        \item Wartosci musza byc liczbowe
        \item Musi istniec co najmniej jeden punkt $t$
    \end{itemize}
\end{enumerate}

\subsection{Przyklady bledow i komunikaty}
\begin{lstlisting}[frame=single, backgroundcolor=\color{gray!10}, basicstyle=\small\ttfamily]
Blad: Brak danych wejsciowych.
Blad: Pierwsza linia musi zawierac liczbe calkowita n.
Blad: n musi byc >= 1.
Blad: Zbyt malo linii danych - oczekiwane: n, x, y.
Blad: Linia 2 musi zawierac wartosci liczbowe x.
Blad: Dla n=2 nalezy podac dokladnie 3 wartosci x (podano 2).
Blad: Wartosci x musza byc unikalne.
Blad: Linia 3 musi zawierac wartosci liczbowe y.
Blad: Dla n=2 nalezy podac dokladnie 3 wartosci y (podano 4).
Blad: W liniach z t musza znajdowac sie liczby.
Blad: Brak punktow t do obliczenia interpolacji.
\end{lstlisting}

\section{Przyklad dzialania}
\subsection{Dane wejsciowe (poprawne)}
\begin{lstlisting}
2
0.0 1.0 2.0
1.0 2.0 5.0
0.5 1.5
\end{lstlisting}

\subsection{Wynik}
\begin{lstlisting}
1.25
3.25
\end{lstlisting}

\subsection{Bledne dane wejsciowe i komunikaty}
\begin{lstlisting}[frame=single, backgroundcolor=\color{gray!10}, basicstyle=\small\ttfamily]
2
0.0 1.0 0.0
1.0 2.0 5.0
0.5 1.5

Blad: Wartosci x musza byc unikalne.
\end{lstlisting}

\begin{lstlisting}[frame=single, backgroundcolor=\color{gray!10}, basicstyle=\small\ttfamily]
2
0.0 1.0 2.0
1.0 2.0
0.5 1.5

Blad: Dla n=2 nalezy podac dokladnie 3 wartosci y (podano 2).
\end{lstlisting}

\begin{lstlisting}[frame=single, backgroundcolor=\color{gray!10}, basicstyle=\small\ttfamily]
2
0.0 1.0 2.0
1.0 2.0 5.0

Blad: Brak punktow t do obliczenia interpolacji.
\end{lstlisting}

\section{Podsumowanie}
\begin{itemize}
    \item Program \textbf{sprawdza poprawnosc danych} przed przystapieniem do obliczen
    \item \textbf{Precyzyjne komunikaty bledow} wskazuja, co jest nie tak
    \item \textbf{Oddzielny modul walidacji} zapewnia czytelnosc kodu
    \item Program \textbf{nie przerywa dzialania} w przypadku bledow - wyswietla komunikat i konczy prace
\end{itemize}

\end{document}